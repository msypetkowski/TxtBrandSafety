\documentclass[a4paper]{article}

\usepackage[a4paper,  margin=1.0in]{geometry}

\usepackage{graphicx}
\usepackage{float}
\usepackage{hyperref}


\usepackage[utf8]{inputenc}
\begin{document}


\title{Text-based brand safety system - realization project}

\author{Mikołaj Ciesielski, Bartosz Paszko, Michał Sypetkowski}
\maketitle

\section{General information}

We are implementing a system
that provides brand safety functionality.

Tools and languages:
\begin{itemize}
    \item \textbf{Python}\footnote{\url{https://www.python.org/}}
    \item \textbf{pykka}\footnote{\url{https://www.pykka.org/en/latest/}}
\end{itemize}



\section{Actor types}
We decided to distinguish 2 types of actors.

\subsection{Main actor}
This actor will accept queries from client applications.
The query will consist of the website body, and the advertisements specification.
Actors of this type will respond for each each advertisement
(whether it conflicts with given website).

\subsection{Text extraction actor}
A website will be represented by it's whole html code.
First, we plan to obtain raw text fragments from it,
then pass these fragments to the text processing actors.

TODO: describe method or mention tools

\subsection{Text processing actor}
Raw text fragments are sent to text processing actors.
Their task is to predict labels for raw text fragments.
They are separated from text extraction actors,
because they will be able to use GPU acceration
for faster processing.

TODO: describe NLP method, find dataset

\end{document}
